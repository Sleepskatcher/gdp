\chapter{Processi Organizzativi}\label{ProcessiOrganizzativi}

\section{Processo di coordinamento}\label{ProcessiOrganizzativiProcessoDiCoordinamento}

\subsection{Scopo}\label{ProcessiOrganizzativiProcessoDiCoordinamentoScopo}
Questa sezione ha lo scopo di definire il metodo di comunicazione uniforme per tutti i componenti del gruppo sia internamente che esternamente. 

\subsection{Descrizione}\label{ProcessiOrganizzativiProcessoDiCoordinamentoDescrizione}
Questa sezione mostra i metodi di coordinamento adottati dal gruppo \textit{Jawa Druids} in termini di riunioni, comunicazione, ruoli del progetto e assegnazione dei compiti. Verranno inoltre brevemente introdotti gli strumenti selezionati e la loro configurazione di base. La struttura del processo di coordinamento secondo lo standard ISO/ IEC 12207 è la seguente:
\begin{itemize}
	\item \textbf{Comunicazione:} interna oppure esterna;
	\item \textbf{Riunioni:} interne oppure esterne.
\end{itemize}

\subsection{Aspettative}\label{ProcessiOrganizzativiProcessoDiCoordinamentoAspettative}
L'aspettativa che il gruppo vuole ottenere è una facilità di comunicazione più elevata e una più efficiente qualità di trasmissione di informazioni tra tutti i componenti del gruppo riducendo i tempi di attesa da risposta e messaggi che possano non essere rilevanti a tutti i componenti del gruppo \textit{Jawa Druids}.

\subsection{Comunicazione}\label{ProcessiOrganizzativiProcessoDiCoordinamentoComunicazione}
Durante il progetto, il team di \textit{Jawa Druids} comunicherà su due diversi livelli: interno ed esterno.
Per quanto riguarda la comunicazione esterna, esse avverranno con i seguenti soggetti:
\begin{itemize}
\item \textbf{Proponente$_{\scaleto{G}{3pt}}$:} l'azienda \textit{Sync Lab}, rappresentata dal signor Fabio Pallaro;
	\item \textbf{Committenti$_{\scaleto{G}{3pt}}$:} nella persona del prof. Tullio Vardanega e del prof. Riccardo Cardin;
		\item \textbf{Competitor:} questo punto verrà chiarito dopo la Revisione dei Requisiti$_{\scaleto{G}{3pt}}$ quando i capitolati saranno assegnati ai relativi gruppi;
			\item \textbf{Esperti interni:} da consultare eventualmente previo accordo con il proponente$_{\scaleto{G}{3pt}}$ ed i committenti$_{\scaleto{G}{3pt}}$.
\end{itemize}
Il gruppo si rivolgerà a tutti i soggetti mediante comunicazioni scritte e/o meeting.
\subsubsection{Comunicazione interna}\label{ProcessiOrganizzativiProcessoDiCoordinamentoComunicazioneInterna}
Il metodo di comunicazione standard per l'interazione scritta tra i membri del gruppo \textit{Jawa Druids} è il servizio di messaggistica istantanea Discord$_{\scaleto{G}{3pt}}$.
La strategia per la gestione delle discussioni sarà anche creare un canale specifico per ogni attività$_{\scaleto{G}{3pt}}$, non ignorabile, che il gruppo deve scegliere (ad esempio, tutte le discussioni sui documenti saranno condotte all'interno di quel canale specifico).
Oltre a questa tipologia di canali, le discussioni saranno suddivise anche nelle seguenti categorie:
\begin{itemize}
\item \textbf{git-github:} per qualsiasi discussione e/o problemi riguardanti la repository$_{\scaleto{G}{3pt}}$;
	\item\textbf{generale:} per qualsiasi discussione generica o riguardante il progetto;
	\item\textbf{\LaTeX:} per qualsiasi discussione riguardante \LaTeX;
	\item\textbf{domande per azienda:} per tutte le domande/ dubbi da porre al proponente$_{\scaleto{G}{3pt}}$.
\end{itemize}
Discord$_{\scaleto{G}{3pt}}$ permette di notificare un particolare messaggio ad una o più persone, includendo nel corpo del messaggio le seguenti parole chiave:
\begin{itemize}
\item \textbf{@everyone:} il messaggio verrà notificato a tutti i componenti del gruppo;
	\item \textbf{@username:} inserendo l'username specifico, il messaggio verrà notificato all'utente desiderato.
\end{itemize}
Inoltre, un'altro modo di comunicazione è la video-chiamata di Discord$_{\scaleto{G}{3pt}}$, che è stato preferito ad altri servizi per la sua versatilità, per il fatto di essere open-source$_G$ e per una essere multi-piattaforma.	
\subsubsection{Comunicazione esterna}\label{ProcessiOrganizzativiProcessoDiCoordinamentoComunicazioneEsterna}
Il Responsabile di Progetto rappresenterà l'intero team e manterrà i contatti esterni tramite un indirizzo email appositamente creato.
L'e-mail utilizzata sarà: 
\begin{center}
	\textbf {JawaDruids@gmail.com}
\end{center}
Il Responsabile di Progetto deve utilizzare regole relative alle comunicazioni interne per notificare a ciascun membro del team eventuali comunicazioni ricevute dal cliente e dal proponente$_{\scaleto{G}{3pt}}$.
Ogni email inviata avrà la seguente struttura:
\begin{itemize}
\item \textbf{Oggetto:} l'oggetto della mail sarà preceduto dalla sigla "[SWE][UNIPD]", in modo che l'ambito di riferimento dell'e-mail sia immediatamente chiaro ed espresso nel modo più conciso possibile per eliminare ambiguità e migliorare la comprensione dell'oggetto;
	\item \textbf{Corpo:} il tono da mantenere è formale, ci si rivolgerà dando del Lei. Il corpo sarà sempre preceduto da una formula di apertura formale, come "Egregio", "Alla cortese attenzione di \textit{Sync Lab}", "Spettabile", a seconda del destinatario.
	Il contenuto dovrà inoltre essere sintetico ed esaustivo, per esporre al meglio il problema e/o eventuali richieste.
\end{itemize}
\subsection{Riunioni}\label{ProcessiOrganizzativiProcessoDiCoordinamentoRiunioni}
Ogni riunione nominerà un segretario il cui compito$_{\scaleto{G}{3pt}}$ è tenere traccia di ciò che viene discusso, eseguire l'ordine del giorno e infine utilizzare le informazioni raccolte per redigere i verbali della riunione.
\subsubsection{Riunioni interne}\label{ProcessiOrganizzativiProcessoDiCoordinamentoRiunioniRiunioniInterne}
Solo i membri del team possono partecipare alle riunioni interne. Almeno quattro persone devono partecipare alla riunione per confermarne la validità. Le decisioni saranno prese a maggioranza semplice ed inoltre, affinché la riunione sia considerata efficace, il responsabile del progetto deve completare le seguenti attività$_{\scaleto{G}{3pt}}$:
\begin{itemize}
\item fissare preventivamente la data della riunione, previa verifica della disponibilità dei membri del gruppo;
\item stabilire un ordine del giorno;
\item nominare un segretario per la riunione.
\end{itemize}
I partecipanti della riunione, invece, devono:
\begin{itemize}
	\item avvertire preventivamente in caso di assenze o ritardi;
	\item presentarsi puntuali al meeting nell'ora prestabilita;
	\item partecipare attivamente alla discussione.
\end{itemize}
Ogni membro del gruppo avrà la possibilità di richiedere un incontro interno.
\subsubsection{Riunioni esterne}\label{ProcessiOrganizzativiProcessoDiCoordinamentoRiunionRiunioniEsterne}
Le riunioni esterne prevedono la partecipazione di soggetti esterni, quali il proponente$_{\scaleto{G}{3pt}}$ e i committenti$_{\scaleto{G}{3pt}}$, oltre ai componenti del gruppo \textit{Jawa Druids}.
Così come le riunioni interne, anche quelle esterne, prevedono la nomina di un segretario che dovrà poi redigere un verbale di riunione.
Le riunioni esterne con il proponente$_{\scaleto{G}{3pt}}$ saranno condotte attraverso il canale Discord$_{\scaleto{G}{3pt}}$ creato appositamente.
\subsection{Strumenti utilizzati per il processo di coordinamento}\label{ProcessiOrganizzativiProcessoDiCoordinamentoStrumentiUtilizzatiPerIlProcessoDiCoordinamento}
\begin{itemize}
	\item \textbf{Discord$_{\scaleto{G}{3pt}}$:} \url{https://discordapp.com/company/};
	\item \textbf{Gmail}: \url{https://www.google.com/intl/it/gmail/about/}.
\end{itemize}
\section{Processo di pianificazione}\label{ProcessiOrganizzativiProcessoDiPianificazione}

\subsection{Scopo}\label{ProcessiOrganizzativiProcessoDiPianificazioneScopo}
L'obiettivo di questa sezione è di definire una descrizione dettagliata di ogni ruolo da svolgere nel periodo di sviluppo del prodotto, definire metriche per la valutazione dei ruoli e gli strumenti utilizzati dai vari ruoli. 


\subsection{Descrizione}\label{ProcessiOrganizzativiProcessoDiPianificazioneDescrizione}
Lo scopo di questa sezione è spiegare come il gruppo intende pianificare il lavoro, dall'assegnazione dei ruoli fino alla concreta assegnazione dei compiti di ogni componente di \textit{Jawa Druids}. In conformità allo standard ISO/IEC 12207, il processo di pianificazione è strutturato nelle seguenti parti:
\begin{itemize}
	\item ruoli di progetto;
	\item assegnazione dei ruoli;
	\item metriche di valutazione;
	\item strumenti.
\end{itemize}

\subsection{Aspettative}\label{ProcessiOrganizzativiProcessoDiPianificazioneAspettative}
Le aspettative che il gruppo vuole ottenere sono:
\begin{itemize}
	\item una definizione di ogni ruolo cosicché ogni componente abbia consapevolezza dei propri compiti;
	\item metriche per la valutazione dello svolgimento delle attività.
\end{itemize}

\subsection{Ruoli di progetto}\label{ProcessiOrganizzativiProcessoDiPianificazioneRuoliDiProgetto}

I vari componenti del gruppo ricopriranno i seguenti ruoli:
\begin{itemize}
	\item \textit{Responsabile di Progetto};
	\item \textit{Amministratore di Progetto};
	\item \textit{Analista};
	\item \textit{Progettista};
	\item \textit{Programmatore};
	\item \textit{Verificatore}.
\end{itemize}

Il gruppo stabilirà un calendario in modo tale che ogni membro riesca a ricoprire almeno una volta ciascuno un ruolo in un periodo di tempo omogeneo senza gravare sullo svolgimento delle attività$_{\scaleto{G}{3pt}}$. Come previsto dal \textit{Piano di Progetto 2.0.0} l'assegnazione di un ruolo comporta lo svolgimento di determinati compiti, inoltre il gruppo si impegnerà per eliminare eventuali conflitti: un componente non potrà mai redigere e successivamente verificare ciò che ha prodotto.

\subsubsection{Responsabile di Progetto}\label{ProcessiOrganizzativiProcessoDiPianificazioneRuoliDiProgettoResponsabileDiProgetto}

Il \textit{Responsabile di Progetto}, ruolo fondamentale e presente per l'intera durata del lavoro, rappresenta il gruppo presso il proponente$_{\scaleto{G}{3pt}}$ ed i committenti$_{\scaleto{G}{3pt}}$. Il suo principale compito$_{\scaleto{G}{3pt}}$ è quello di coordinare la struttura ed organizzare il lavoro. In particolare questo ruolo comporta:
\begin{itemize}
	\item il coordinamento dei membri del gruppo e dei compiti che devono portare a termine;
	\item la gestione della pianificazione, ossia l'attività$_{\scaleto{G}{3pt}}$ da svolgere e le relative scadenze da rispettare;
	\item avere la responsabilità della stima dei costi e dell'analisi dei rischi;
	\item la gestione delle relazioni che il gruppo intrattiene con i soggetti esterni;
	\item l'amministrazione delle risorse umane e dell'assegnazione dei ruoli;
	\item l'approvazione dei documenti.
\end{itemize}

\subsubsection{Amministratore di Progetto}\label{ProcessiOrganizzativiProcessoDiPianificazioneRuoliDiProgettoAmministratoreDiProgetto}

L'\textit{Amministratore del Progetto} ha il compito$_{\scaleto{G}{3pt}}$ di gestire, controllare e curare gli strumenti che il gruppo utilizza per lo svolgimento del proprio lavoro; è colui che garantisce l'affidabilità e l'efficacia dei mezzi. Questa figura persegue l'idea che la buona gestione dell'ambiente del lavoro favorisca la produttività, per questo motivo deve:
\begin{itemize}
	\item amministrare le infrastrutture e i servizi necessari ai processi di supporto;
	\item gestire il versionamento e la configurazione dei prodotti;
	\item controllare la documentazione per assicurarsi che venga corretta, verificata ed approvata;
	\item facilitare il reperimento della documentazione;
	\item risolvere eventuali problemi legati alla gestione dei processi;
	\item redigere e manutenere le norme e le procedure$_{\scaleto{G}{3pt}}$ che regolano il lavoro;
	\item individuare gli strumenti utili all'automazione dei processi.
\end{itemize}

\subsubsection{Analista}\label{ProcessiOrganizzativiProcessoDiPianificazioneRuoliDiProgettoAnalista}

L'\textit{Analista} è presente nelle fasi iniziali del progetto, soprattutto quando avviene la stesura dell'\textit{Analisi dei Requisiti} e il suo compito$_{\scaleto{G}{3pt}}$ è quello di evidenziare i punti fondamentali del problema in questione per comprenderne le sue peculiarità. Quindi la sua figura è fondamentale per la buona riuscita del lavoro, in quanto senza la sua analisi potrebbero essere presenti errori o mancanze nell'individuazione dei requisti che possono compromettere la successiva attività$_{\scaleto{G}{3pt}}$ di progettazione.\\
L'\textit{Analista}:
\begin{itemize}
	\item studia e definisce il problema in oggetto;
	\item analizza le richieste;
	\item fissa quali sono i requisiti$_{\scaleto{G}{3pt}}$ studiando i bisogni impliciti ed espliciti;
	\item analizza gli utenti e i casi d'uso;
	\item redige lo \textit{Studio di Fattibilità} e l'\textit{Analisi dei Requisiti}. 
\end{itemize}

\subsubsection{Progettista}\label{ProcessiOrganizzativiProcessoDiPianificazioneRuoliDiProgettoProgettista}

Il \textit{Progettista} ha il compiti di sviluppare una soluzione che soddisfi i bisogni individuati dal lavoro dell'\textit{Analista}. Lo scopo di questo compito$_{\scaleto{G}{3pt}}$, di natura sintetica, è quello di produrre un'architettura che modelli il problema a partire da un insieme di requisiti$_{\scaleto{G}{3pt}}$. Egli deve:
\begin{itemize}
	\item applicare i principi noti e collaudati per produrre un'architettura coerente e consistente;
	\item produrre una soluzione sostenibile e realizzabile che rientri nei costi stabiliti dal preventivo;
	\item costruire una struttura che soddisfi i requisiti$_{\scaleto{G}{3pt}}$ e che sia aperta alla comprensione;
	\item limitare il più possibile il grado di accoppiamento tra le varie componenti;
	\item sforzarsi di cercare l'efficienza, la flessibilità e la riusabilità;
	\item elaborare una soluzione capace di interagire in modi diversi con l'ambiente in cui si pone e che sia sicura rispetto ad eventuali anomalie e intrusioni esterne;
	\item ricercare la massima disponibilità e affidabilità per l'architettura proposta.
\end{itemize}

\subsubsection{Programmatore}\label{ProcessiOrganizzativiProcessoDiPianificazioneRuoliDiProgettoProgrammatore}

Il \textit{Programmatore} è incaricato della codifica: il suo compito$_{\scaleto{G}{3pt}}$ è quello di implementare l'architettura prodotta dal \textit{Progettista} in modo tale che aderisca alle specifiche. Egli è responsabile della manutenzione del codice creato, infatti i suoi compiti sono:
\begin{itemize}
	\item codificare secondo le specifiche dettate dal \textit{Progettista}, inoltre il codice prodotto deve essere documentato, versionabile e strutturato così da agevolare la futura manutenzione;
	\item creare le componenti che servono per la verifica e la validazione del codice;
	\item scrivere il \textit{Manuale Utente} relativo al prodotto.
\end{itemize}

\subsubsection{Verificatore}\label{ProcessiOrganizzativiProcessoDiPianificazioneRuoliDiProgettoVerificatore}

Il \textit{Verificatore} deve essere presente per tutta la durata del progetto e si occupa di controllare che le attività$_{\scaleto{G}{3pt}}$ svolte rispettino le norme e le attese prefissate. Egli deve:
\begin{itemize}
	\item accertarsi che l'esecuzione delle attività$_{\scaleto{G}{3pt}}$ di processo non provochi errori;
	\item redigere la parte retrospettiva del \textit{Piano di Qualifica} che chiarisce le verifiche e le prove effettuate.
\end{itemize}

\subsection{Assegnazione dei compiti}\label{ProcessiOrganizzativiProcessoDiPianificazioneAssegnazioneDeiCompiti}

La progressione del progetto può essere vista come il completamento sequenziale o parallelo di una serie di compiti, ognuno con scadenza temporale, i quali producono risultati utili per la realizzazione degli obiettivi. I compiti possono essere determinati da:
\begin{itemize}
	\item contingenza;
	\item processi in atto;
	\item un insieme dei fattori precedenti.
\end{itemize}

La figura che si occupa della suddivisione ed assegnazione dei compiti è il \textit{Responsabile di Progetto}, il quale:
\begin{itemize}
	\item individua il compito$_{\scaleto{G}{3pt}}$ da svolgere;
	\item se ritiene il compito$_{\scaleto{G}{3pt}}$ complesso lo suddivide in diversi sotto-compiti;
	\item individua uno o più componenti del gruppo a cui assegnare il compito$_{\scaleto{G}{3pt}}$;
	\item crea le schede su Trello$_G$ e aggiorna la Timeline di GitKraken$_{\scaleto{G}{3pt}}$.
\end{itemize}

Di conseguenza i membri del gruppo devono impegnarsi per svolgere il compito$_{\scaleto{G}{3pt}}$ entro la data prefissata, avvisando nel caso in cui riscontrino problemi a rispettare le scadenze.

\subsection{Trello e Gitkraken}\label{ProcessiOrganizzativiProcessoDiPianificazioneTrelloEGitkraken}

Dopo aver suddiviso i compiti, il \textit{Responsabile di Progetto} modificherà la pagina di Trello$_{\scaleto{G}{3pt}}$ del gruppo. Ogni scheda avrà la descrizione del compito$_{\scaleto{G}{3pt}}$ da svolgere, i componenti del gruppo che devono svolgerlo e la data di scadenza.
Una volta che il task$_G$ è concluso viene spostata la sua scheda relativa nella sezione "Need revision" ovvero "Bisogno di revisione".
Svolta l'attività$_{\scaleto{G}{3pt}}$ di verifica la scheda viene spostata in "Need approval", ossia che ha bisogno di approvazione, e infine in "Completed", ovvero completato.\\
Durante lo sviluppo i membri del gruppo possono aggiungere commenti ed informazioni utili allo svolgimento del compito$_{\scaleto{G}{3pt}}$. Se quest'ultimo è suddiviso in più parti sarà presente un elenco puntato che indica ogni suddivisione che l'addetto allo svolgimento potrà spuntare quando conclusa. \\
Per aiutare il gruppo viene fornita anche una rappresentazione grafica delle varie scadenze da rispettare grazie allo strumento Timeline di GitKraken$_{\scaleto{G}{3pt}}$.

\subsection{Metriche}\label{ProcessiOrganizzativiProcessoDiPianificazioneMetriche}

\subsubsection{MQPS01 Budget at Completion}\label{ProcessiOrganizzativiProcessoDiPianificazioneMetricheMQPS01BudgetAtCompletion}
Budget totale allocato per il progetto secondo il preventivo riportato nel \textit{Piano di Progetto 2.0.0}.
Si possono verificare tre casi:
\begin{itemize}
	\item il budget preventivato è minore di quello effettivo al momento della valutazione, questo ci può indicare due cose:
	 \begin{itemize}
	 	\item[-] il preventivo è stato sottostimato e quindi va rivisto;
	 	\item[-] il metodo di lavoro è sbagliato e c’è uno spreco di risorse;
	 \end{itemize}
 	\item il budget preventivato e quello effettivo corrispondono, presumibilmente si sta lavorando nella maniera corretta;
 	\item il budget preventivato è maggiore di quello effettivo, questo nella maggior parte dei casi indica che il preventivo è stato fatto con un margine troppo ampio.
\end{itemize}

\subsubsection{MQPS02 Planned Value}\label{ProcessiOrganizzativiProcessoDiPianificazioneMetricheMQPS02PlannedValue}
Metrica di utilità per il calcolo della SV (spiegata successivamente).
Si tratta del valore del lavoro pianificato al momento del calcolo: corrisponde al denaro che
si dovrebbe aver guadagnato in quel momento.
La formula adottata è:  
	\begin{center}
		PV = BAC * \textit{\% lavoro pianificato}
	\end{center}
\begin{itemize}
	\item \textbf{PV} sigla di \textit{Planned Value};
	\item \textbf{BAC} sigla di \textit{Budget at Completion}, riportato sopra;
	\item \textit{\% lavoro pianificato} al momento del calcolo.
\end{itemize}

\subsubsection{MQPS03 Actual Cost}\label{ProcessiOrganizzativiProcessoDiPianificazioneMetricheMPQS03ActualCost}
Il denaro speso fino al momento del calcolo per lo svolgimento del progetto.
E’ molto importante misurare periodicamente questo valore in modo da riuscire a rendersi conto dell'andamento dei costi durante lo svolgimento del progetto e quindi di non superare la soglia del budget totale prima del completamento dello stesso.
Per il calcolo si devono sommare tutti i costi già sostenuti.

\subsubsection{MQPS04 Earned Value}\label{ProcessiOrganizzativiProcessoDiPianificazioneMetricheMQPS04EarnedValue}
Metrica di utilità per il calcolo di SV e CV (spiegate successivamente).
Si tratta del valore del lavoro fatto fino al momento del calcolo;
corrisponde al denaro guadagnato fino a quel momento.

\subsubsection{MQPS05 Schedule Variance}\label{ProcessiOrganizzativiProcessoDiPianificazioneMetricheMQPS05ScheduleVariance}
Indica lo stato di avanzamento nello svolgimento del progetto rispetto a quanto pianificato.
La formula adottata è: 
	\begin{center}
		SV = EV - PV
	\end{center}
\begin{itemize}
	\item \textbf{SV} è la sigla di \textit{Schedule Variance};
	\item \textbf{EV} è la sigla di \textit{Earned Value}, cioè il valore effettivo di quanto prodotto alla data corrente;
	\item \textbf{PV} è la sigla di \textit{Planned Value}, cioè il valore programmato per la data corrente.	
\end{itemize}

\subsubsection{MQPS06 Cost Variance}\label{ProcessiOrganizzativiProcessoDiPianificazioneMetricheMQPS06CostVariance}
Indica la differenza tra il costo di lavoro effettivamente completato e quindi “ribaltabile” al cliente come valore,  ed il costo attualmente sostenuto.
La formula adottata è:
	\begin{center}
		CV = EV - AC
	\end{center}
\begin{itemize}
	\item \textbf{CV} sigla di \textit{Cost Variance};
	\item \textbf{EV} sigla di \textit{Earned Value}, riportato sopra;
	\item \textbf{AC} sigla di \textit{Actual Cost}, riportato sopra.
\end{itemize}


\subsection{Strumenti}\label{ProcessiOrganizzativiProcessoDiPianificazioneStrumenti}
Per il supporto alla pianificazione del progetto e alla realizzazione dei diagrammi di Gannt$_G$ il gruppo ha deciso di utilizzare Fogli Google$_{\scaleto{G}{3pt}}$, software multipiattaforma con le seguenti funzionalità:
\begin{itemize}
	\item la possibilità di modificare in maniera collaborativa e contemporanea i fogli di lavoro;
	\item la possibilità di creare diagrammi da tabelle.
\end{itemize}

\section{Processo di miglioramento} \label{ProcessiOrganizzativiProcessoDiMiglioramento}
\subsection{Scopo} \label{ProcessiOrganizzativiProcessoDiMiglioramentoScopo}
Il processo di miglioramento ha come scopo quello di assicurare il miglioramento del processo di sviluppo del prodotto, attraverso misurazioni e monitoraggi. 
\subsection{Descrizione} \label{ProcessiOrganizzativiProcessoDiMiglioramentoDescizione}
Il processo di miglioramento è impiegato per la manutenzione migliorativa dei processi durande lo sviluppo del prodotto software. Ciò si effetua tramite attività di valutazione e miglioramento.  
\subsection{Aspettative} \label{ProcessiOrganizzativiProcessoDiMiglioramentoAspettative}
Tramite tale sezione il gruppo si impegna a comprendere l'importanza del miglioramento continuo dei processi e dell'applicazione di tale processo. 
\subsection{Valutazione del processo}\label{ProcessiOrganizzativiValutazioneDelProcesso}
Durante lo sviluppo del prodotto software, il gruppo deve adottare un procedimento di valutazione di processo, da applicare ad ogni processo, in modo tale che ciascuno di questi sia sviluppato e monitorato costantemente. E' necessario tenere traccia, attraverso la documentazione, di tale attività di valutazione, apportando delle modifche ai processi se ritenuto necessario. 
Per poter implementare in modo corretto e continuativo questo procedimento, per ogni periodo, verrà riportata, nel \textit{Piano di Qualifica}, una tabella contenente le seguenti informazioni per ogni processo:
\begin{itemize}
	\item i problemi riscontrati;
	\item la descrizione di ciascun problema;
	\item il livello di gravità attribuito ad ogni problema, attraverso un indice numerico che varia da 1 a 3;
	\item la soluzione individuata dal gruppo per risolvere il problema.
\end{itemize}

\section{Formazione}\label{ProcessiOrganizzativiFormazione}
\subsection{Scopo}\label{ProcessiOrganizzativiFormazioneScopo}
L'obiettivo di questa sezione è di definire riferimenti per ogni tecnologia e struttura utilizzate durante lo svolgimento del progetto, tali riferimenti verranno utilizzati per la formazione personale al fine di ottenere un personale qualificato ed esperto.
\subsection{Descrizione}\label{ProcessiOrganizzativiFormazioneDescrizione}
All'interno di questa sezione vengono indicate le varie ducumentazioni dove è possibile studiare le tecnologie e strutture per lo svolgimento di documenti e progetto. Inoltre sono presenti anche riferimenti a guide sull'utilizzo degli strumenti utili per l'organizzazione del gruppo e sviluppo del prodotto.
\subsection{Aspettative}\label{ProcessiOrganizzativiFormazioneAspettative}
Il gruppo vuole ottenere attraverso questa sezione la formazione di ogni componente in una persona esperta e qualificata nello svolgimento di ogni suo compito e utilizzo di ogni strumento. Questa formazione permette così di ottenere un prodotto di qualità.
\subsection{Documentazione}\label{ProcessiOrganizzativiFormazioneDocumentazione}
I membri del gruppo devono provvedere alla propria formazione autonoma studiando le tecnologie usate e colmando eventuali carenze, per poter garantire una qualità di lavoro che rispetti le aspettative. Il gruppo fa riferimento alla seguente documentazione e a quella elencata nei riferimenti dei documenti redatti:
\begin{itemize}
	\item \textit{\LaTeX:} \url{https://www.latex-project.org/} e \url{https://www.overleaf.com/learn/latex/Main_Page};
	\item \textit{\TeX Studio:} \url{https://www.texstudio.org/};
	\item \textit{Vue.js$_G$:} \url{https://vuejs.org/};
	\item \textit{Spring$_G$:} \url{https://spring.io/};
	\item \textit{Java$_G$:} \url{https://docs.oracle.com/en/java/};
	\item \textit{Leaftletjs$_G$:} \url{https://leafletjs.com/reference-1.7.1.htm};
	\item \textit{Mqtt$_G$:} \url{https://mqtt.org/};
	\item \textit{Scikit-learn$_G$:} \url{https://scikit-learn.org/stable/index.html};
	\item \textit{Python$_{\scaleto{G}{3pt}}$:} \url{https://www.python.org/doc/};
	\item \textit{Kafka$_G$:} \url{https://kafka.apache.org/documentation/};
	\item \textit{Github$_G$:} \url{https://help.github.com/};
	\item \textit{Git$_{\scaleto{G}{3pt}}$:} \url{https://git-scm.com/};
	\item \textit{Git-flow$_{\scaleto{G}{3pt}}$:} \url{https://github.com/nvie/gitflow};
	\item \textit{Farfalla Project$_G$:} \url{https://farfalla-project.org/readability_static/}.
\end{itemize}