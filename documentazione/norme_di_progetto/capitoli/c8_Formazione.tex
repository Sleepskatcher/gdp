\chapter{Formazione}\label{Formazione}
\section{Processo di formazione}\label{FormazioneProcessoDiFormazione}
Tramite il processo di formazione si intendono fornire materiali e strumenti idonei a rendere il gruppo di lavoro qualificato allo sviluppo del prodotto software.
Questo processo consiste nelle seguenti attività$_{\scaleto{G}{3pt}}$:
\begin{enumerate}
	\item \textbf{Piano di formazione};
	\item \textbf{Ricerca del materiale};
	\item \textbf{Inizio della formazione}.
\end{enumerate}
\subsection{Piano di formazione}\label{FormazioneProcessoDiFormazionePianoDiFormazione}
In questa prima fase il gruppo di lavoro, assieme al \textit{Responsabile di progetto}, discuteranno e decideranno su quali siano le abilità principali e necessarie per diventare sviluppatori qualificati nell'ambito del prodotto software.
Questa decisione sarà basata sulle richieste proposte nel capitolato d'appalto$_{\scaleto{G}{3pt}}$ del committente$_{\scaleto{G}{3pt}}$.
\subsection{Ricerca del materiale}\label{FormazioneProcessoDiFormazioneRicercaDelMateriale}
Una volta definito il piano da seguire si prosegue con la ricerca attiva del materiale da parte del gruppo.
Questo lavoro può essere svolto mediante tre macro-categorie:
\begin{itemize}
	\item \textbf{Libri}: quindi la ricerca tramite libri testuali, o digitali, contenenti nozioni sugli argomenti da imparare;
	\item \textbf{Azienda}: ovvero la richiesta di materiale specifico all'azienda proponente$_{\scaleto{G}{3pt}}$ del software da sviluppare;
	\item \textbf{Internet}: quindi cercando su forum o siti specializzati informazioni pertinenti tutto ciò che riguarda il progetto.
\end{itemize}
\subsection{Inizio della formazione}\label{FormazioneProcessoDiFormazioneInizioDellaFormazione}
Raccolto il materiale adatto, il gruppo inizierà la propria formazione sia personale, sia collettiva nel caso qualche componente avesse problemi con alcuni concetti.
