\newpage
\section{Informazioni Generali}
\begin{enumerate}
  \item \textbf{Luogo:} \normalfont Server Discord \textbf{Jawa Druids};
  \item \textbf{Data:} \normalfont 08-03-2021;
  \item \textbf{Orario inizio:} \normalfont 16.30;
  \item \textbf{Orario fine:} \normalfont 17.30;
  \item \textbf{Partecipanti:}
  \begin{itemize}
    \item Mattia Cocco;
    \item Andrea Dorigo;
    \item Margherita Mitillo;
    \item Igli Mezini;
    \item Andrea Cecchin;
    \item Emma Roveroni;
    \item Alfredo Graziano.
  \end{itemize}
  \item \textbf{Assenti:}
  \begin{itemize}
    \item Nessuno
  \end{itemize}
\end{enumerate}
\section{Ordine del giorno}
Una volta che tutti membri del gruppo di lavoro erano pronti è iniziata la riunione. L'incontro si è incentrato sui segenti punti:
\begin{itemize}
  \item aggiornamento sullo sviluppo del software per la TB;
  \item organizzazione per la presentazione della TB.
\end{itemize}

\section{Aggiornamento sullo sviluppo del software per la TB}
Ogni \textit{modulo} ha fatto vedere, mediante condivisione schermo, gli aggiornamenti effettuati dall'ultimo incontro interno.
Inoltre sono state fatte delle prove, collegandosi alla VM, in modo da essere pronti a per esporlo al \textit{Prof. Riccardo Cardin} il giorno 11-03-2021 alla TB.
Tutti i \textit{moduli} sono risultati collegati tra loro e funzionanti, eccetto il \textit{modulo 4}, del machine learning, che è ancora sviluppato a parte per mancanza dati reali nel database.


\section{Organizzazione per la presentazione della TB}
In questi giorni è stata fatta, inoltre, la presentazione da esporre alla TB.
Ogni \textit{modulo} ha redatto un piccolo discorso per il lavoro fatto così da spiegare le tecnologie utilizzate e le righe di codice fulcro del proprio lavoro.
Infine verrà fatto vedere il funzionamento del software al docente.


\section{Decisioni derivate dal colloquio}
  \begin{itemize}
    \item \textbf{I\_08-03-2021.1}: continuare coi test nei prossimi giorni in modo da avere una mole di dati abbastanza importante;
    \item \textbf{I\_08-03-2021.2}: ultimare la presentazione ed imparare il proprio discorso, anche mediante prove collettive per rimanere all'interno dei 15 minuti di tempo concessi.
  \end{itemize}
