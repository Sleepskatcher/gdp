\newpage
\section{Informazioni Generali}
\begin{enumerate}
  \item \textbf{Luogo:} \normalfont Server Discord \textbf{Jawa Druids};
  \item \textbf{Data:} \normalfont 10-02-2021;
  \item \textbf{Orario inizio:} \normalfont 16.30;
  \item \textbf{Orario fine:} \normalfont 17.30;
  \item \textbf{Partecipanti:}
  \begin{itemize}
    \item Mattia Cocco;
    \item Andrea Dorigo;
    \item Margherita Mitillo;
    \item Igli Mezini;
    \item Andrea Cecchin;
    \item Emma Roveroni;
    \item Alfredo Graziano.
  \end{itemize}
  \item \textbf{Assenti:}
  \begin{itemize}
    \item Nessuno
  \end{itemize}
\end{enumerate}
\section{Ordine del giorno}
Una volta che tutti membri del gruppo di lavoro erano pronti è iniziata la riunione. L'incontro si è incentrato sui segenti punti:
\begin{itemize}
  \item punto della situazione sulla documentazione da riconsegnare;
  \item gestione dello sviluppo del progetto.
\end{itemize}

\section{Punto della situazione sulla documentazione da riconsegnare}
Il gruppo ha verificato le modifiche apportate ai documenti bloccati alla valutazione della \textit{Revisione dei Requisiti}, per valutarne la qualità.


\section{Gestione dello sviluppo del progetto}
Dopo un confronto il gruppo ha deciso di suddividere il progetto in \textit{moduli} e assegnarli ai vari componenti.
In questo modo si facilita il tracciamento degli avanzamenti per ciascun modulo e di conseguenza semplifica sia la leggibilità del codice che i collegamenti tra i moduli.


\section{Decisioni derivate dal colloquio}
  \begin{itemize}
    \item \textbf{I\_10-02-2021.1}: continuare a modificare l'\textit{Analisi dei Requisiti} per migliorarla ulteriormente;
    \item \textbf{I\_10-02-2021.2}: inizio, da parte dei membri del gruppo, della codifica dei moduli assegnati.
  \end{itemize}
