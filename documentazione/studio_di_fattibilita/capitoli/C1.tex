\chapter{C1 - BlockCOVID: supporto digitale al contrasto della pandemia} \label{CapitolatoC1}

Il capitolato$_{\scaleto{G}{3pt}}$ C1 è stato presentato da \textit{Imola Informatica}, azienda che si occupa di consulenza IT.

\section{Informazioni generali} \label{C1InformazioniGenerali}
\begin{itemize}
	\item \textbf{Nome} - BlockCOVID: supporto digitale al contrasto della pandemia;
	\item \textbf{Proponente}$_{\scaleto{G}{3pt}}$ - \textit{Imola Informatica};
	\item \textbf{Committente}$_{\scaleto{G}{3pt}}$ - Prof. Tullio Vardanega e Prof. Riccardo Cardin.
\end{itemize}
\section{Descrizione del capitolato} \label{C1DescrizioneDelCapitolato}
L'azienda proponente$_{\scaleto{G}{3pt}}$ ha deciso di trattare la tematica della pandemia contemporanea in ambito alla sicurezza al lavoro. Infatti ogni azienda dovrebbe assicurare ai propri dipendenti un luogo di lavoro sicuro dal rischio di contagio e, perciò, sanificato correttamente. In particolare bisogna poter segnalare le postazioni in uso e , successivamente, comunicare quando vengono liberate in modo che gli addetti possano procedere con una sanificazione corretta, rendendo la postazione pronta per un nuovo utilizzo.\\
Questa procedura, oltre a tutelare il dipendente, tutela anche il datore di lavoro nel caso in cui avvenga un caso di contagio.
\section{Finalità del progetto} \label{C1FinalitàDelProgetto}
L'obiettivo è quello di sviluppare un'applicazione in grado di indicare quando una postazione viene occupata da un determinato dipendente. In particolare, tramite l'applicazione, si deve poter sapere se una postazione è libera, occupata oppure prenotata, sapere lo stato di avanzamento della sanificazione e prenotare una postazione. Gli utenti che possono usare questa applicazione sono divisi in tre categorie: amministratore, utente ed addetto alle pulizie. Il primo deve poter gestire le postazioni di lavoro, i dipendenti presenti ed estrapolare un report legato alle postazioni utilizzate da un singolo utente ed uno legato alle sanificazioni effettuate. Il secondo, invece, deve poter prenotare e segnalare l'occupazione della postazione e quando la pulisce con il kit aziendale. Il terzo, infine, deve poter ricevere un elenco delle postazioni che necessitano di sanificazione e spuntare quelle sanificate.
\section{Tecnologie interessate} \label{C1TecnologieInteressate}
Il proponente$_{\scaleto{G}{3pt}}$ preferisce non imporre particolari tecnologie da utilizzare per svolgere il progetto in quanto sempre interessato alla ricerca di nuove soluzioni tecnologiche. L'azienda si sente comunque di consigliare al fornitore le seguenti tecnologie: 
\begin{itemize}
	\item \textit{Java}$_{\scaleto{G}{3pt}}$, \textit{Python} o \textit{nodeJS} per lo sviluppo del Back-End$_{\scaleto{G}{3pt}}$.;
	\item \textit{IAAS Kubernetes} oppure un \textit{PAAS} per il rilascio delle componenti.
\end{itemize}
\section{Aspetti positivi} \label{C1AspettiPositivi}
\begin{itemize}
	\item La possibilità di scegliere autonomamente le tecnologie da utilizzare.
\end{itemize}
\section{Criticità e fattori di rischio} \label{C1CriticitàEFattoriDiRischio}
\begin{itemize}
	\item Il prodotto software di questo capitolato$_{\scaleto{G}{3pt}}$ è legato ad una tematica ristretta e che quindi difficilmente potrà essere applicato in campi diversi da quello in cui nasce;
	\item Block Covid è un capitolato$_{\scaleto{G}{3pt}}$ che sviluppa un'applicazione con idee già esistenti e nella teoria semplici da capire che non porta il gruppo ad un percorso di autoformazione.
\end{itemize}
\section{Conclusioni} \label{C1Conclusioni}
Questo capitolato$_{\scaleto{G}{3pt}}$ non ha suscitato particolare interesse nel gruppo in quanto legato ad un tema troppo ristretto. Inoltre non si è riusciti ad evidenziare un possibile percorso di autofromazione riguardo a argomenti diversi da quelli del proprio percorso di studi. Per questo motivo, il gruppo ha preferito orientarsi verso un'alternativa più stimolante.