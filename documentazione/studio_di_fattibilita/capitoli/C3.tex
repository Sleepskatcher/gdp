\chapter{Capitolato scelto: C3 - GDP, Gathering Detection Platform} \label{CapitolatoC3}

Il capitolato$_{\scaleto{G}{3pt}}$ C3 è stato presentato dall'azienda \textit{Sync Lab}, Software House  nata nel 2002 che propone nel mercato prodotti software nei settori mobile, videosorveglianza e sicurezza nelle infrastrutture e informatiche aziendali. L’obiettivo di \textit{Sync Lab} è la realizzazione, messa in opera e governance di soluzioni IT. Inoltre è molto sensibile all'innovazione attraverso attività di ricerca  e sviluppo.

\section{Informazioni generali} \label{C3InformazioniGenerali}
\begin{itemize}
	\item \textbf{Nome} - GDP: Gathering Detection Platform;
	\item \textbf{Proponente}$_G$ - \textit{Sync Lab};
	\item \textbf{Committente}$_G$ - Prof. Tullio Vardanega e Prof. Riccardo Cardin.
\end{itemize}
\section{Descrizione del capitolato} \label{C3DescrizioneDelCapitolato}
La pandemia dovuta al  virus Covid-19 ha portato noi cittadini inizialmente ad una quarantena forzata, successivamente ad una parziale circolazione. Questo ha comportato alla creazione di situazioni di rischio assembramento e di conseguenza ad un aumento del  pericolo di contagio. L'idea dell'azienda è quella di creare quindi una piattaforma che possa aiutare i cittadini a vivere più serenamente e con sicurezza questa situazione. Infatti, attraverso tale servizio si potrà individuare quali zone sono più a rischio assembramento rispetto ad altre. \\
Per realizzare ciò, è necessario utilizzare elementi raccolti da sensoristica e da altre sorgenti, in modo tale da poter fare una stima ed ottenere indicazioni sui potenziali aggregamenti e conseguentemente fornire un supporto per l'ottimizzazione del traffico pedonale.
\section{Finalità del progetto} \label{C3FinalitàDelProgetto}
La soluzione al problema precedentemente descritto deve essere un prototipo software in grado di acquisire e monitorare dati per poi estrapolarne le informazioni da sfruttare in modo da identificare le zone e/o  eventi che presentino un rischioso flusso di persone. I fruitori della piattaforma devono poter quindi visualizzare dati in tempo reale tramite heat map$_G$ oppure predizioni future di una determinata zona. \\
In particolare gli obiettivi tecnologici che si vogliono raggiungere sono:
\begin{itemize}
	\item Realizzazione di un software atto a contare le persone nei mezzi pubblici;
	\item Realizzazione di simulazione dei dati per poter monitorare dati storici e previsionali;
	\item Capacità di acquisire informazioni a bassa latenza ed in modo continuativo;
	\item Elaborazione in tempo reale dei dati;
	\item Identificazione di eventi concorrenti;
	\item Previsione di variazione del flusso di utenti.
\end{itemize}
Riguardo al lato predittivo si dovrà istruire il software tramite Machine Learning$_G$ a riconoscere dati di momenti passati ed elaborare predizioni future. Inoltre bisogna aggiornare automaticamente le previsioni sulla base dei nuovi dati che vengono osservati.
\section{Tecnologie interessate} \label{C3TecnologieInteressate}
Il proponente$_{\scaleto{G}{3pt}}$ ha interesse nell'esplorare nuove tecnologie quindi preferisce non imporne di specifiche, affidandosi alle proposte dei fornitori. Ha specificato comunque alcune scelte tecnologiche da considerare per lo svolgimento del progetto:
\begin{itemize}
	\item \textit{Java}$_G$ e \textit{Angular}$_G$ per lo sviluppo delle parti di Back-End$_G$ e Front-End$_G$;
	\item Il framework$_G$ \textit{Leaflet}$_G$ per la gestione delle mappe.
\end{itemize}
\section{Aspetti positivi} \label{C3AspettiPositivi}
\begin{itemize}
	\item La possibilità di lavorare ad un progetto legato a tematiche contemporanee;
	\item Anche se il progetto nasce da un determinato tema attuale, il gruppo ha concluso che un prodotto software di questo tipo ha applicazione in diversi campi, quindi vi è una possibilità di utilizzo dei ragionamenti e dei metodi di sviluppo in futuro;
	\item La possibilità di non avere vincoli tecnologici;
	\item Il proponente$_{\scaleto{G}{3pt}}$ è aperto al confronto e disponibile a creare un ambiente caratterizzato da una forte collaborazione;
	\item La possibilità di esplorare tecnologie non presenti nel percorso di studi universitario;
	\item Interesse da parte di tutti i componenti del gruppo a lavorare con il Machine Learning$_{\scaleto{G}{3pt}}$.
\end{itemize}
\section{Criticità e fattori di rischio} \label{C3CriticitàEFattoriDiRischio}
\begin{itemize}
	\item L'apprendimento delle nuove tecnologie o delle strumentazioni previste potrebbe risultare lento per i membri del gruppo che non le hanno mai utilizzate.
\end{itemize}
\section{Conclusioni} \label{C3Conclusioni}
Il capitolato$_{\scaleto{G}{3pt}}$ ha attirato l'attenzione del gruppo fin da subito grazie alla chiarezza e alla linearità della presentazione. La possibilità di avere libertà tecnologica e la propensione alla ricerca di tecnologie innovative da parte del proponente$_{\scaleto{G}{3pt}}$ potrebbe permetterci di imparare argomenti non presenti negli insegnamenti del percorso di studi. La presenza del Machine Learning$_{\scaleto{G}{3pt}}$ è stato un altro fattore decisivo per la scelta del capitolato$_{\scaleto{G}{3pt}}$, settore interessante che non viene toccato nel percorso di studi. A seguito di queste affermazioni, il gruppo \textit{Jawa Druids} ha eletto questo capitolato$_{\scaleto{G}{3pt}}$ come prima scelta, fiduciosi del fatto di riuscire a colmare le lacune e affrontare in sinergia ogni possibile difficoltà che si presenterà nel processo di creazione del prodotto software richiesto.